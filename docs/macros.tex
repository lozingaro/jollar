\usepackage{bold-extra}

\newcommand{\Jolie}{Jolie}
\newcommand{\Definition}{\noindent\textbf{\emph{Definition}}}
\newcommand{\Implementation}{\noindent\textbf{\emph{Implementation}}}
\newcommand{\Section}{\S}

\newcommand{\citeNeed}{{\color{red}[CitNeed]}}

\definecolor{color:keyword}{rgb}{0.53,0.05,0.05}
\definecolor{color:comment}{rgb}{0.25,0.37,0.75}
\definecolor{color:string}{rgb}{0.87,0.0,0.0}

\lstdefinelanguage{Jolie}{
    morekeywords={csets,type,raw,any,undefined,void,default,if,for,while,spawn,foreach,else,define,main,include,constants,inputPort,outputPort,interface,execution,cset,nullProcess,RequestResponse,OneWay,throw,throws,install,scope,embedded,init,synchronized,global,is_defined,is_int,is_bool,is_long,is_string,bool,long,int,string,double,undef,with,Location,Protocol,Interfaces,Aggregates,Redirects,linkIn,linkOut},
    sensitive=true,
    morecomment=[l]{//},
    morecomment=[s]{/*}{*/},
    morestring=[b]",
    otherkeywords={;,|,@}
}

\lstset{
    language=Jolie,
    mathescape=true,
    resetmargins=true,
    numberstyle=\footnotesize,
    numbers=none,
    numbersep=5pt,
    numberblanklines=true,
    basicstyle=\ttfamily\small,
    tabsize=2,
    %frame=lines,
    commentstyle=\rmfamily\color{color:comment},
    stringstyle=\color{color:string},
    captionpos=b,
    keywordstyle=\bfseries\color{color:keyword},
    showstringspaces=false,
    belowcaptionskip=10mm,
    breaklines=false,
    columns=fullflexible,
    linewidth= 0.8\linewidth
}

\newcommand{\code}[1]{\lstinline{#1}{}}

\newcommand{\setUmlSeqChartStyle}{
    \tikzset{inststyle/.style={
    rectangle, draw, 
    anchor=west, 
    minimum height=0.8cm, 
    minimum width=1.6cm, 
    fill=white
    %drop shadow={opacity=0,fill=black}]
    }
  }
}

\renewcommand{\mess}[4][0]{
  \stepcounter{seqlevel}
  \path
  (#2)+(0,-\theseqlevel*\unitfactor-0.7*\unitfactor) node (mess from) {};
  \addtocounter{seqlevel}{#1}
  \path
  (#4)+(0,-\theseqlevel*\unitfactor-0.7*\unitfactor) node (mess to) {};
  \draw[->,>=angle 60] (mess from) -- (mess to)% 
    node[midway, above, align=center, text width=3cm]
    {\footnotesize #3};

  \node (#3 from) at (mess from) {};
  \node (#3 to) at (mess to) {};
}

\crefname{figure}{Fig.}{Figs.}
\crefname{lstlisting}{Listing}{Listings}
