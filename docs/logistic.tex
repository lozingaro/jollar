In questa sezione sono riportate le informazioni logistiche a riguardo del progetto: un \textit{reminder} sulla formazione dei gruppi; un elenco di possibili date per la consegna dell'implementazione e del report. Tali informazioni sono soggette a cambiamenti ed a revisioni, ogni modifica viene comunicata attraverso la pagina ufficiale del corso, la pagina del tutor, ed il \href{https://www.wikibooks.org}{newsgroup}. È possibile chiedere delucidazioni e prenotare un ricevimento via messaggio di posta elettronica, specificandone la motivazione, a \href{mailto:stefanopio.zingaro@unibo.it}{questo indirizzo}.

\subsection{Formazione dei Gruppi}
I gruppi sono costituiti da un minimo di tre ad un massimo di cinque persone, coloro che intendono partecipare all'esame comunicano entro e non oltre il \textbf{10 Maggio 2018} (pena esclusione dall'esame) la composizione del gruppo di lavoro, \textbf{via posta elettronica}, all'indirizzo \url{stefanopio.zingaro@unibo.it}. Il messaggio ha come oggetto \textbf{GRUPPO LSO} e contiene:
\begin{enumerate}
    \item Il nome del gruppo;
    \item Una riga per ogni componente: cognome, nome e matricola;
    \item Un indirizzo di posta elettronica di riferimento a cui mandare le notifiche, è incarico del proprietario trasmetterle agli altri membri.
\end{enumerate}
\begin{tcolorbox}[colback=green!20!white,colframe=green!75!black,title=Email di esempio con oggetto \textbf{GRUPPO LSO}]
\textit{NomeGruppo}
\begin{itemize}
    \item \textit{Vader Darth, 123456}
    \item \textit{Pallino Pinco, 234567}
    \item \textit{Banana Joe, 345678}
\end{itemize}
\textit{Referente: joe.banana@studio.unibo.it}
\end{tcolorbox}

Chi non riuscisse a trovare un gruppo invia allo stesso indirizzo di posta elettronica un messaggio con oggetto \textbf{CERCO GRUPPO LSO}, specificando:
\begin{enumerate}
    \item Cognome, Nome, Matricola, Email;
    \item Eventuali preferenze legate a luogo e tempi di lavoro (si cercherà di costituire gruppi di persone con luoghi e tempi di lavoro compatibili)
\end{enumerate}
\begin{tcolorbox}[colback=green!20!white,colframe=green!75!black,title=Email di esempio con oggetto \textbf{CERCO GRUPPO LSO}]
\textit{Pallino Pinco, 234567, preferirei nei pressi del dipartimento, tutti i giorni dopo pranzo.}
\end{tcolorbox}
Le persone senza un gruppo vengono assegnate il prima possibile senza possibilità di ulteriori modifiche.

\subsection{Date di consegna e dell'Orale}
 Ci sono due date disponibili per la consegna dell'implementazione e del report. Queste sono:
\begin{itemize}
    \item le 23.59 di \textbf{Lunedí 2 Luglio} 2018;
    \item le 23.59 di \textbf{Lunedí 17 Settembre} 2018.
\end{itemize}
La data presa in considerazione per la consegna della parte di implementazione sarà quella di creazione del \textbf{Tag} su \href{https://gitlab.com}{GitLab} (le istruzioni più avanti nel testo). Solo in seguito alle consegne, vengono fissate data ed orario della discussione (notificate tramite la mail di riferimento), compatibilmente coi tempi di correzione.
La discussione dell'implementazione e la relativa demo di funzionamento viene effettuata in un incontro unico con tutti i componenti presenti. Al termine della discussione, ad ogni singolo componente verrà assegnato un voto in base all'effettivo contributo dimostrato nel lavoro. La valutazione è indipendente dal numero di persone che compongono il gruppo.
