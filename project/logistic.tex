\subsection{Formazione dei Gruppi}
%
I gruppi possono essere costituiti da un minimo di 3 a un massimo di 5 persone. 
I gruppi che intendono svolgere questo progetto, devono comunicare via email a \url{stefanopio.zingaro@unibo.it} entro il \textbf{10 Maggio 2018} la composizione del gruppo. 
L'email deve avere come oggetto \textbf{GRUPPO LSO} e contenere:
\begin{enumerate}
    \item Nome del gruppo;
    \item Una riga per ogni componente del gruppo, con cognome, nome e matricola;
    \item Un indirizzo email di riferimento a cui mandare le notifiche al gruppo, sarà poi suo incarico trasmetterle agli altri membri.
\end{enumerate}
Email di esempio con oggetto \textbf{GRUPPO LSO}

\noindent\textit{NomeGruppo}
\begin{itemize}
    \item \textit{Zingaro, Stefano, 123456}
    \item \textit{Pallino, Pinco, 234567}
    \item \textit{Banana, Joe, 345678}
\end{itemize}
\textit{Referente:} \url{stefano.zingaro@studio.unibo.it}

Chi non comunicherà la composizione del gruppo entro il \textbf{10 Maggio 2018} non potrà consegnare il progetto. 
Nel caso, chi non riuscisse a trovare un gruppo, lo comunichi il prima possibile, entro e non oltre il \textbf{7 Maggio 2018}, a \url{stefano.zingaro@studio.unibo.it} con una mail con oggetto \textbf{CERCO GRUPPO LSO}, specificando:
\begin{enumerate}
    \item Cognome, Nome, Matricola, Email;
    \item Eventuali preferenze legate a luogo e tempi di lavoro (si cercherà di costituire gruppi di persone con luoghi e tempi di lavoro compatibili)
\end{enumerate}
Email di esempio con oggetto \textbf{CERCO GRUPPO LSO} 
\begin{itemize}
    \item \textit{Zingaro, Stefano, 123456,} \url{stefano.zingaro@studio.unibo.it}
\end{itemize}
\textit{Preferisco trovarci nei pressi del dipartimento, tutti i giorno dopo pranzo.}

Le persone senza un gruppo verranno raggruppate il prima possibile \textbf{e non sarà possibile modificare i gruppi formati}.
%
\subsection{Date dell'Orale}
%